% You should title the file with a .tex extension (hw1.tex, for example)
\documentclass[11pt]{article}

\usepackage{amsmath}
\usepackage{amssymb}
\usepackage{fancyhdr}


\oddsidemargin0cm
\topmargin-1.5cm     %I recommend adding these three lines to increase the
\textwidth16.5cm   %amount of usable space on the page (and save trees)
\textheight23.5cm

\newcommand{\question}[2] {\vspace{.25in} \hrule\vspace{0.5em}
\noindent{\bf #1: #2} \vspace{0.5em}
\hrule \vspace{.10in}}
\renewcommand{\part}[1] {\vspace{.10in} {\bf (#1)}}

\newcommand{\myname}{Laxman Dhulipala}
\newcommand{\myandrew}{ldhulipa@andrew.cmu.edu}
\newcommand{\myhwnum}{5}

\setlength{\parindent}{0pt}
\setlength{\parskip}{5pt plus 1pt}

\pagestyle{fancyplain}


\begin{document}

\medskip                        % Skip a "medium" amount of space
                                % (latex determines what medium is)
                                % Also try: \bigskip, \littleskip

\thispagestyle{plain}
\begin{center}                  % Center the following lines
{\Large Coresets} \\
\myname \\
\myandrew
\end{center}

\newcommand{\betaskel}[0]{$\beta$-skeleton}

\section{Clustering}

A central theme throughout Har-Peled's work is the idea of efficient approximation. Even if a problem appears
to be computationally intractable, it's likely that we can find an approximation of some sort. However, let's 
first take a look at some general clustering problems first. 

Intuitively, a cluster is a set of objects, $S$, from a given set of objects, $P$, that are more `similar' 
in some way within $S$ than to objects outside of $S$. In particular, the objects we will examine are points
typically in $\mathbb{R}^{d}$, with similarity generally being captured by distance, or minimizing the total 
weight of points in a cluster. 

Har-Peled presents his definition similarly : Given an input point set, $P$, where $P$ is a part of a metric
space with distance function $d$, we can, given another set of centers, $C$, cluster every point of $P$
to its nearest neighbour in $C$. 




\end{document}
